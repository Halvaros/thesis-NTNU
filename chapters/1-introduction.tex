\chapter{Introduction}

On the background in traffic modelling
Problemstilling: 
The present paper aims at examining the connection between to modelling paradigms for traffic flows. It has been established that there exist a connection between microscopic follow-the-leader (FtL) model and the classic Lighthill and Whitham model, in the so-called many particle limit. FtL models vehicular traffic dynamics using a discrete set of vehicle-driver units that populate and propagate on a unidirectional road. As the number of particles increase and the length of each car decreases in such a way that preserves the total mass, the countable system of ODEs converge to a hyperbolic partial differetial equation, which can be viewed as a certain type of uncountable collection of ODEs. Partial differential equations are a generalisation of ODEs in several dimensions. To be precise, we concern ourselves with the problem of finding pairs (FtL, SCL) such that

\begin{equation}\label{FtL}
    \frac{dx_n}{dt} = k(x_n)v(1/s_n). -> \rho_t + f(\rho,x)_x = 0,
\end{equation}
in some appropriate sense. If such a relation can be established, the microscopic model can be seen as a numerical approximation to the macroscopic approach. The theory behind the two modelling paradigms are very different, and such a connection is of independent interest. The motivation behind this approach is to examine This has been done before -> link to papers.

pairs of the (FtL,This has been shown in several papers, for example. This papers examines a particular (FtL) model on the form, 
The complete dynamics of each unit $x_n$ is governed an ordinary differential equation, depending on the bumper-to-bumper distance $s_n$ of the closest leading vehicle. Where $k(x)$ can be any suitably chosen smooth function.  [HelgePaper] shows that 

ach These can be the bumper-to-bumper distance of proceding cars, the speed of the driver unit in the presence of a speed and so on. limit and varying other factors depending on the situation and complexity of the model. One model of intereset is the simple (X) model,
For finitely many 

\textcite{argall2002rigorous}

The Lighthill and Whitham model takes the form of a scalar conservation law,
\begin{equation}
    \rho_t + f(\rho)_x = 0
\end{equation}
- Intelligent driver etc. 

For the special system, 

\begin{equation} \label{special inhom SCL}
    \rho_t + (k(x)f(\rho))_x = 0
\end{equation}
where $k \in C^1[\R]$, we derive the entropy condition. Introduce a viscosity solution that satisifies $\rho^{\epsilon}_t + (k(x)f(\rho^{\epsilon}))_x = \epsilon  \frac{\partial^2 \rho^\epsilon}{\partial^2x} $ and consider any convex function $\xi \in C^1$ differentiable a.e. (fact from analysens grunnlag any convex function is differentiable almost everywhere), and multiply by the equation by the $\xi^{'}(\rho^{\epsilon})$. This leads to the equation 

\begin{equation}
    \xi(\rho^{\epsilon})_t + k(x) \xi^{'}(\rho^{\epsilon}) f^{'}(\rho^{\epsilon})\rho^{\epsilon}_x + k^{'}(\rho^{\epsilon})(x)\xi^{'}f(\rho^{\epsilon}) = \epsilon \xi^{\rho}(\rho^{\epsilon}) \frac{\partial^2 \rho^\epsilon}{\partial^2x}. 
\end{equation}
Note that $\partial_x(\xi \partial_x \rho) = \xi^{''} \partial_x \rho^2 + \xi^{'}\partial^2_{xx}\rho$ implies that the right hand term is less than $\partial_x(\epsilon \xi(\rho^{\eps}) \partial_x \rho)$. Assume $\epsilon \to 0$, then 

\begin{equation}
    \xi(u)_t + (k(x)q(u))_x + k^{'}(x)\big( \xi^{'}(\rho^{\epsilon}) f(\rho^{\epsilon})  q(\rho^{\epsilon})\big).
\end{equation}
Let $\xi^\delta (u) = \sqrt{(u-k)^2 + \delta^2}$, then $\partial_u \xi^\delta = \frac{u - k}{\sqrt{(u - k}^2 + \delta ^2} \to \sgn(u - k)$ pointwise. By the dominated convergence theorem, 
where There is done research on the case where 


On hyperbolic conservation laws: 
On conservation laws in their most general form. - Dafermos. 
On the L1-contractitivty of scalar conservation law. 
What came before. 

Kruzkow entropy solution, weak solution. 

History of microscopic and macroscopic models: 
- Two different approaches to traffic modelling are microscopic and macroscopic modelling. Microscopic models are density based and can be derived as continuum equations. Kinetic gass formulation.  These have been studied extensively in recent years. 

- How they differ, central recults. 

The motivation for this notion of admissibility is
provided by the observation that all the systems of balance laws from Continuum
Physics encountered in Chapter II are indeed accompanied by some inequality
in the form (4.3.3) which expresses, explicitly or implicitly, the Second Law of
thermodynamics. - Dafermos 

\section{Some mathematical definitions}
Here are some definitions that will be used throughout the thesis. 
\begin{definition}[Function]
A function is a rule that assigns to each input value to a unique output value. For sets $X$ and $Y$, we write 
\begin{equation}
    f : X \rightarrow Y, 
\end{equation}
and say that $f$ denotes the function between $X$ and $Y$. A set is a collection of distinct object. Examples of sets include the set of real numbers, denoted by $\mathscr{R}$, and $\mathscr{R}^n$, the $n$-dimensional euclidean space. \\
\end{definition}

A class of function that is of interest are continuous functions. Continuity for functions between metric spaces, like $\R$, is most commonly first introduced in the $\epsilon - \delta$-fashion: 
\begin{definition}[Continuity between metric spaces]
A function  $f : X \rightarrow Y$ between metric spaces $X$, $Y$ is continuous at $x \in X$ if 
given $\epsilon > 0 \,\, \exists \,\, \delta > 0 s.t. \forall y \in X$ where
\begin{equation}
    d(x,y) < \delta \implies d(f(x),f(y)) < \epsilon.
\end{equation}
\end{definition}
}

Geometrically, for each $\epsilon$ ball around an image point, there exists at least on $\delta-ball$ such that the image of the latter is contained in the former. A continuous function is continuous at every point $x \in X$. The set, or space of continuous functions betweeen $X$ and $Y$ is denoted by $\C(X,Y)$. Let $\Omega \subset \mathscr{R}^n$. One important class is

\begin{equation} \label{C(Rn)}
     \mathscr{C}(\Omega) := \{f : \Omega \rightarrow \mathscr{R} \,|\, \text{f is a continuous function\}.
\end{equation}
If $\Omega$ is compact, a result by SOMEONE, states that every element of $\C$ attains its supremal value, and can be endowed with the supremum norm. Often called the topology of uniform convergence. 

\begin{equation}
    \norm{f}_\infty = sup_{x \in \Omega} |f(x)|.
\end{equation}
By (...), is a banach space, or a complete normed space. 

\begin{align*}
    \mathscr{C}(\mathscr{R}^n) := \{f : \mathscr{R}^n \rightarrow \mathscr{R} \,|\, \text{f is continuous at every x $\in \mathscr{R}^n$}\}, \\
    \mathscr{C}^{(k)}(\mathscr{R}^n) := \{f : \mathscr{R}^n \rightarrow \mathscr{R} \,|\, \textit{f is $k$-times continuously differentiable}\} \subset \mathscr{C}(\mathscr{R}^n) \, \forall k \in \mathscr{N}, \\
    \mathscr{C}^{\infty}(\mathscr{R}^n) := \{f : \mathscr{R}^n \rightarrow \mathscr{R} \,|\, \frac{\partial^{\alpha} f}{\partial^{\alpha} x}\in \mathscr{C}(\mathscr{R}^n) \, \forall \, \alpha \in \mathscr{Z}^n\} = \bigcap_{i=0}^{\infty} \mathscr{C}^{(k)}(\mathscr{R}^n).
\end{align*}
There are several equivalent definitions and generalisations of continuity, the standard one being a local definition. 
USE DEFINITIONS 

Total variation is a measure of how much a function changes. For a function that takes pointwise values, we have the following definition

\subsection{Functions of bounded variation and Lipschitz continuous functions}
Some results and definition concerning functions of bounded variation and Lipschitz continuous functions. Functions of bounded variation permeate the theory of hyperbolic conservation laws, and is often a vital assumption to make for the solutions to obtain existence and uniqueness theory. As we will see in the main part of the thesis, they serve a vital role. 

Lipschitz continuous funtions are an important subclass of functions of bounded variation. In many cases, as good as differentiable. Many important results in differentiable calculus extend to Lipschitz continuous functions. 

\begin{definition}(Bounded variation in one dimension)
Let $f : \R \rightarrow \R$ be a function. The \emph{total variation} T.V.($f$) of $f$ is defined to be the supremum
\begin{equation} \label{TV_1d}
	\text{T.V.}(f) = \sup_{x_0 < ... < x_n} \sum_{i = 0}^{n-1} \left| f(x_{i+1}) - f(x_i)\right|
\end{equation} 
where the supremm ranges over all finite increasing sequences $x_0, ..., x_n$ of real numbers with $n \geq 0$; this is a quantity in $[0, \infty]$. We say that $f$ has \emph{bounded variation} on $\R$ if T.V.($f$) is finite. If T.V.($f$) is finite when $x_0, ..., x_n$ are restricted to a compact set $K$, we say that $f$ is of \emph{locally bounded variation}. Other ways to write T.V.($f$) include $\left|f\right|_{\text{BV}}$ to emphasise the semi-norm property. %Book that proves that BV norm is a banach space
%BV functions are bounded 
%

A result from the theory of bounded variations concerns the decomposition of B.V-functions. 

\begin{proposition} \label{TV_monotone_decomp}
	A function $f : \R \mapsto \R$ is of bounded variation if and only if it is the difference of two bounded monotone functions. A monotone function is a function that is either increasing or decreasing.
\end{proposition}
For proof, \cite[see][p.166]{tao2011introduction}.
This allows the following corollary
\begin{corollary}
	Any function $f : \R \mapsto \R$ of bounded variation has a left- continuous version $\tilde{f}}$ such that 
	\begin{equation}
		\text{T.V.}(f) = \text{T.V.}(\tilde{f})
	\end{equation}
\end{corollary}
\begin{proof}
	By \eqref{TV_monotone_decomp}, $f$ can be decomposed into a sum of two bounded monotone functions. For a monotone function $g$, the existence of left continuous limit is equivalent to the statement
	\begin{align}
		\lim_{y \rightarrow x^-} g(y) \text{ exists } \forall x \in \R.  \label{montone_left_right_lim1} 
	\end{align}
	 To see \eqref{montone_left_right_lim1}, assume without loss of generality that $g$ is monotone and increasing. Let $\{x_n^+\}_{n \in \N}$ be a sequence approaching $x$ from below. $g(x_n^+)$ is is a bounded increasing sequence, by assumption, and  convergent to some left limit $g^+(x)$ which is independent of  $\{x_n^+\}_{n \in \N}$. If it was not independent, one could define two sequences $\{x_n\}_{n \in \N}, \{y_n\}_{n \in \N}$ such that $\lim_{n \in \N} x_n = \lim_{n \in \N} y_n = x$ from below,  such that $\lim_{n \in \N} g(x_n) < \lim_{n \in \N} g(y_n)$. Then, $\exists N$ such that $ g(x_n) < g(y_n) \forall n \geq \N$. Fix $N_1 \geq N$ and pick $N_2$ such that $x_{N_2} > y_{N_1}$. This violates the assumption that $g$ is increasing.
	 
	 \begin{equation}
	 	\lim_{y \rightarrow x^-} g(y) \leq g(x) \leq \lim_{y \rightarrow x^+} g(y) \label{montone_left_right_lim2}
	 \end{equation}
 	Implies that the total variation of $\lim_{y \rightarrow x^-} g(y)$ is that of $g$. $\lim_{y \rightarrow x^+} g(y)$ exist by the same argument as given above. For both monotone functions that compose $f$, one can therefore pick left continuous versions for both.
 	
\end{proof}
The notion of T.V.$(\cdot)$ can be extended to functions $f:\R^n \mapsto \R$ for $n \geq 2$. One way of doing this is by taking the one dimensional total variation with respect to each component, as follows

\begin{definition}(One dimensional bounded variation in two dimensions)
	Let $f : \R\times\R \rightarrow \R$ be a function where $x_1, x_2$ are the variables of $f$. Define the function 
	
	\begin{align} \label{TV_slice}
		\text{T.V.}_{x_1}(f) : \R &\mapsto \R^+_0 \cup \infty \\ 	
		x &\mapsto \text{T.V.}(f(\cdot, x)) \nonumber
	\end{align}
	 For each $x \in \R$, $\text{T.V.}_{x_1}(f)$ returns the one dimensional total variation of $f$ restricted to the line $\{x_2 = x\} \subset \R^2$.  $\text{T.V.}_{x_2}(f)$ can be define similarly. 
\end{definition} 

\begin{remark}
	From here, T.V.($f$) can be extended to $\R^2$ by
	\begin{equation} \label{TV_2d}
		\text{T.V.}(f) = \int_\R \text{T.V.}_{x_1}(f) + \text{T.V.}_{x_2}(f) \diff x
	\end{equation}
	   If $f$ is differentiable, \eqref{TV_2d} is proportional to $\norm{\nabla g}_{L_1}$. In this thesis, \eqref{TV_slice} will mainly be used. 
\end{remark}
	%Hva med locally of bounded variation?
	
	
	The \emph{total variation} T.V.($f$) of $f$ is defined to be the supremum
	\begin{equation}
		\text{T.V.}(f) = \sup_{x_0 < ... < x_n} \sum_{i = 0}^{n-1} \left| f(x_{i+1}) - f(x_i)\right|
	\end{equation} 
	where the supremm ranges over all finite increasing sequences $x_0, ..., x_n$ of real number with $n \geq 0$; this is a quantity in $[0, \infty]$. We say that $f$ has \emph{bounded variation} on $\R$ if T.V.($f$) is finite. Other ways to write T.V.($f$) include $\left|f\right|_{\text{BV}}$ to emphasise the semi-norm property.

In this thesis, we will denote 

An important subclass of locally B.V. functions are the Lipschitz continuous functions. 

\begin{definition}(Lipschitz continuous functions over $\R$)
	A function $f : \R \mapsto \R$ is Lipschitz continuous if $ \exists C \in \R^+_0$ such that $\forall x,y \in \R$
	
	\begin{equation} \label{def:Lipschitz_R}
		\left|f(x) - f(y)\right| \leq C\left|x - y\right|
	\end{equation} 
	A function is said to be locally Lipschitz if given $K \subset \R$ closed and bounded, then $\exists C_K$ such that $\forall x,y \in K$
	\begin{equation} \label{def:Lipschitz_K}
		\left|f(x) - f(y)\right| \leq C_K\left|x - y\right|
	\end{equation} 
	The smallest $C$ (resp. $C_K$)such the the above properties is known as the Lipschitz constant $L$ (resp. $L_K$) of $f$. 
\end{definition} 

To see that Lipschitz continuous functions are of locally bounded variation, observe that for any partition 

\begin{equation}
	\sum_{i = 0}^{n-1} \left| f(x_{i+1}) - f(x_i)\right| \leq L\left| x_0 - x_n \right| \leq L \cdot \text{diam}(K)
\end{equation}
So if $K$ is bounded, then T.V.($f$) $< \infty$. In particular, if $f$ is defined on a compact set, then it is of bounded variation. 

\begin{proposition}(Lipschitz differentiation theorem) \label{prop:Lip_diff_prop}
	A Lipschitz continuous function is differentiable almost everywhere, in the sense of Lebesgue measure. Where the derivative exists, it is bounded by the Lipschitz constant $L$. 
\end{proposition}
\begin{proof}
	For a statement of the theorem, see \cite[p.167]{tao2011introduction}. This follows from the fact that Lipschitz functions are locally of bounded variation. 
\end{proof}

\eqref{prop:Lip_diff_prop} states that Lipschitz functions are almost differentiable. 

Two properties of Lipschitz continuous functions that will be useful are 
\begin{proposition}(Lipschitz properties) \label{prop: Lipschitz_props}
	Let f,g : $\R \mapsto \R$ be Lipschitz continous. Then the composition
	\begin{equation}
		f \circ g : \R \mapsto \R
	\end{equation}
	is Lipschitz continuous. 
	
	If f,g are also bounded, then the product
	\begin{equation}
		f g : \R \mapsto \R
	\end{equation}
	is Lipschitz continuous. Furthermore, let $f', g'$ be the almost everywhere derivatives of $f, g$. Then the product rule for differentiation  
	\begin{equation}
		(fg)' = f'g + fg'
	\end{equation}
	holds almost everywhere.
\end{proposition}

Even for continuous B.V. functions, it is not true that the difference in function values can found by integrating the almost everywhere derivative. If the variation of the function is concentrated on a set of lebegue measure zero, the second fundamental theorem of calculus will fail to hold\footnote{see The Devil's staircase function ~\cite[p. 186]{tao2011introduction}}}. The following theorem shows that this does not occur for Lipschitz continuous functions. 

\begin{theorem}(Second fundamental theorem of calculus for Lipschitz functions)
	
	Let f : $\R \mapsto \R$ be Lipschitz continuous and $a,b \in \R$. Then, 
	\begin{equation}
		\int_a^b f(x) dx = f(b) - f(a).
	\end{equation}
\end{theorem}
\begin{proof}
	Exercise 1.6.44 in ~\autocite[p.169]{tao2011introduction}
\end{proof}

\subsection{Compactness for partial differential equations}

Central in the theory of non-linaer partial differential equations is the quest for compactness. One can often formulate an approximating sequence of functions with the property such that the partial differential equation is satisfied in the limit. For the porous medium equation, this can be done by using the semi-discretisation and a forward difference scheme in the spatial variables, as can be seen in. To solve scalar conservation laws and system of conservation laws, this strategy is also proposed. Compactness results are needed to establish that the approximating sequence itself converges to some candidate solution, which then must be verified to satisify the original equation. The compactness trick is in many ways the substance of the argument. Therefore, I will introduce three notions of compactness that will be used in this thesis implicitly and explicitly. The relevant setting is that of metric spaces. 

\begin{definition}(Metric space)
	A \emph{metric space} is a pairing of a set $X$ and a function $d$,  
	\begin{align}
		d : X \times X \mapsto \R^+_0, 
	\end{align}
	$d$ satisifies the metric axioms. For $x,y,z \in X$, 
	\begin{align}
		d(x,y) &= 0 \iff x = y \,\, &&\text{identity of indiscernibles} \\
		d(x,y) &= d(y,x) \ \,\, &&\text{symmetry of distances} \\
		d(x,y) &= d(x,z) + d(z,y)  \,\, &&\text{subadditivity}
	\end{align}
	An open ball $B_x(r) \subset X$ centered at $x \in C$ with radius $r > 0$ is
	the set of points with distance less than $r$ from $x$ according to the metric $d$. 
\end{definition}

A definition of topology is also included 

\begin{definition}(Topology and the metric topology)
	
	Let $X$ be a set. A \emph{topology} $\tau \subset 2^X$ is any collection of subsets of $X$ that is closed under arbitary union and finite intersection. $\tau$ must also include $\varnothing$ and the space $X$ itself. 
	
	From any collection of subsets $C$ of $X$, one can generate a topology. This is done by adding $\varnothing$ and $X$ to $C$, if necessary, and expanding the collection over arbitrary unions and finite intersection. 
	
	Let $(X,d)$ be a metric space. The \emph{metric topology} $\tau_d \subset 2^X$ is the topology generated by open balls $B_x(r)$ for $x \in X$ and $r > 0$. 
\end{definition}

Examples of metric spaces include $\R^d$ endowed with the euclidean metric and $L^1(\R)$. 

\begin{definition} (Completeness in metric spaces)
	A metric space $(X,d)$ is complete if every cauchy sequence has a limit in $X$. 
\end{definition}



Sequential compactness for metric spaces is defined as follows

\begin{definition}(Sequential compactness for metric spaces)
	
	Let $(X,d)$ be a metric space and $K \subset X$ be a subset. $K$ is \emph{sequentially compact} if for any sequence $\{x_n\}_{n \in \N} \subset K$, there exist a subsequence $\{x_{n_i}\}_{i \in \N}$ converging to some $\hat{x} \in K$. 
	If every sequence of $K$ has a convergent subsequence, but we cannot guarantee that the limit lies in $K$, then $K$ is \emph{sequentially precompact}.
\end{{definition}

A topological definition of compactness uses the the open cover concepts. 

\begin{definition} (Open covers and topological compactness)
	
	Let $(X,\tau)$ be a topological space. Let $K\subset X$ be a subset. An open cover of $K$ is a (possibly uncountable) collection of open sets $C \subset \tau$ such that 
	\begin{equation}
		K \subset \cup_{O \in C} O.
	\end{equation}
	A refinement of the open cover $C$ is an open cover $\tilde{C}$ of $K$ such that 
	\begin{equation}
		K \subset \cup_{O \in \tilde{C}} O  \text{ and } \tilde{O} \in \tilde{C} \implies \exists O \in C \text{ such that } \tilde{O} \subset O. 
	\end{equation}
	A subset $K$ is said to be \emph{topologically compact}, or simply \emph{compact} if an arbitrary open cover admits a finite refinement. That is, a refinement consisting of a finite number of open set. 
\end{definition}

\begin{remark}
 	If $X = \R$ is endowed with the euclidean metric, a subset $K$ is compact if and only if it is closed and bounded. Pre-compactness is thus equivalent with boundedness. %CITE
 	The notions introduced in this section can thus be seen as generalisations of the boundeness concept in one dimension. Other generalisations are aplenty, as can be seen in %CITE. 
\end{remark}

The following theorem concept important for characterising compact metric spaces

\begin{definition} (Totally bounded sets)
	
	Let $(X,d)$ be a metric space. $X$ is \emph{totally bounded} if for every $\epsilon > 0$, $\exists x_1, ..., x_n \in X$ for $n \in \N$ such that 
	
	\begin{equation}
		X \subset \cup_{i = 1}^n B_{x_i}(\epsilon).
	\end{equation} 
	$B_{c}(\epsilon)$ is the open ball of radius $\epsilon$, centered at $c$. Such a covering is dubbed an $\epsilon$-net.
\end{definition}

This concept resembles the definition of topological compactness, in that it stipulates the existence of finite open covers, the $\epsilon$-nets. The three concepts of this section are in fact identical for complete metric spaces.

\begin{theorem} (Equivalence theorem)
	
	1) A metric space is topologically compact, or compact, if and only if it is totally bounded and complete. 
	
	
	2) A metric space is compact if and only if it is sequentially compact. 
	
\end{theorem}

The Kolmogorov-Riesz-Sudakov theorem uses the concept of total boundedness to establish conditions for strong $L^p$-compactness. The concept of sequential compactness is useful to apply the theory, when compactness has already been established. For this thesis, we will use sequential compactness directly and the other concepts implicitly, e.g. by invoking of Arzela-Ascoli and the strong compactness criteria of Kolmogorov-Riesz-Sudakov. 





\subsection{Some theory of $L^p$ spaces}

We define, as in ~\autocite{brezis2010functional}
\begin{definition} (Measure space)
	
	A measure space is a triple $(\Omega, \mathcal{M}, \mu)$. 
	\begin{align}
		1) \,&\Omega \text{ is a set.} \\
		2) \,&\mathcal{M} \text{ is a collection of subsets of $\Omega$ such that:} \\
		&\quad \quad i)\, \varnothing \in \mathcal{M},   \\
		&\quad \quad ii) \,A \in \mathcal{M} \implies A^c \in \mathcal{M}  \\
		&\quad \quad iii) \,\cup_{i = 1}^\infty A_n \in \mathcal{M} \text{ whenever } A_n \in \mathcal{M}\\ 
		&\, \mathcal{M} \text{ is called a \emph{$\sigma$-algebra}. The sets of $\mathcal{M}$ are measurable sets}\\
		3) \,& \mu \text{ is a \emph{measure}, meaning a function } \mu : \mathcal{M} \mapsto \R^0_+, \text{ such that:} \\ 
		&\quad \quad i)\, \mu(\varnothing) = 0, \\
		&\quad \quad ii)\, \mu\left(\cup_{i = 1}^\infty A_n\right) = \sum_{i = 1}^\infty \mu\left( A_n \right) \text{ for a disjoint countable collection $\left(A_n\right)$ of measurable sets.} \\
		4) \,&\Omega \text{ is $\sigma$-finite. There exists a countable collection $\Omega_n \in \mathcal{M}$} \\
		&\text{such that $\Omega = \cup_{i = 1}^\infty \Omega_n$ and $\mu\left(\Omega_n\right) < \infty$.} 
	\end{align}                                                                 
\end{definition}
From integration theory, we know that a function need not be defined everywhere to be differentiable. ~\autocite[compare][80]{landes1951scrutiny} \citetext{landes1951scrutiny}
 \nocite{*} 
 

Measure theory introduces a generalisation of the function definition, and considers equivalence classes of functions. 

\begin{definition}(The almost everywhere equivalence class)
	Let $f,g : \Omega \mapsto \R$ be two functions defined on a measure space $(\Omega, \Sigma, \mu)$.  Let $N = \{x \in \Omega \, | \, f(x) \neq g(x)) \}$. $f,g$ are equal almost everywhere if 
	\begin{equation}
		\mu(N) = 0.
	\end{equation}
	We write 
	\begin{equation}
		[f] = [g]. 
	\end{equation}
\end{definition}

Furthermore, we define a measurable function

\begin{definition}(Measurable function)
	Let $(\Omega, \Sigma, \mu)$ be a measure space. 
\end{definition}



\subsection{Results from differentiable calculus and discrete difference rules}
We will however need some results that require the full regularity of continous differentiability. A fundamental result in differentiable calculus is the mean value theorem

\begin{theorem}(The mean value theorem for $\C^{(1)}(\R)$-functions)
	For any $f \in \mathscr{C}([a,b]) \cap \mathscr{C}^{(1)}((a,b))$ we have that 
	
	\begin{equation} \label{MVT}
		\frac{f(b) - f(a)}{b-a} = \frac{df}{dx}(c)
	\end{equation}
	for some $c \in (a,b)$. As a function of the endpoints, $c$ is continuous. 
\end{theorem}
\begin{proof}
	Found in \cite[p.142]{finney2000calculus}, using Rolle's theorem. 
\end{proof}


The following result concern functions  differentiation under the integral sign. 

\begin{theorem}(Leibniz rule for differentiation under the integral)
	Let  $f:\mathscr{R} x \mathscr{R} \rightarrow \mathscr{R}$ be a differentiable function such that $\frac{\partial f}{\partial t}(x,t)$
	is continuous on $\R\times\R$. Suppose that $a, b \in \C^{(1)}(\R)$. Then
	\begin{equation} \label{LeibRule}
		\frac{d}{dt}\int_{a(t)}^{b(t)} f(x,t) dx = - f(a(t),t) \frac{da}{dt}(t) + f(b(t),t) \frac{db}{dt}(t) + \int_{a(t)}^{b(t)}\frac{\partial f}{\partial t}(x,t) dx. 
	\end{equation}
\end{theorem}
Formula is taken from 
\begin{proof}
	See \cite[p.255]{kaplan1912advanced}. 
\end{proof}


A generalisation of the mean-value formula for a closed interval $[a,b]$ reads:
\begin{equation}\label{genMVT}
	\int_a^bf(x)g(x)dx = f(c) \int_a^b g(x) dx, c \in (a,b), 
\end{equation}
where $f \in \mathscr{C}([a,b])$ and $g \in L_1([a,b])$, in the sense of lebesgue measure. 

A Grönwall type estimate.
\newtheorem{theorem}[T
heorem]

\begin{theorem}[Da theorem]
	This statement is true, I guess.
\end{theorem}

\begin{lemma}{A simple Grönwall type estimate}
	\begin{equation}
		\frac{df}{dt} \leq A + B f \Rightarrow f(t) \leq \left( f(0) + \frac{A}{B}\right)e^{B t} - \frac{A}{B}.
	\end{equation}
\end{Lemma}
This can be proved using integrating factors and integrating on both sides. 
\begin{align}
	\frac{df}{dt} &\leq A + B f  \nonumber \\
	\Rightarrow \frac{d \left(f e^{-Bt}\right)}{dt} &\leq A e^{-Bt}\nonumber \\
	\Rightarrow e^{-Bt}f(t) - f(0) &\leq \frac{A}{B}\left(1 - e^{-Bt}\right)\nonumber \\
	\Rightarrow f(t) &\leq \left( f(0) + \frac{A}{B}\right)e^{B t} - \frac{A}{B}. \qed
\end{align}
The preceding theorem holds for any absolutely continuous function, in particular it will hold for a Lipshchitz continuous functions. 

The following formula is very simple, but will be used extensively. It is the formula for splitting the terms of a difference 

\begin{formula}(Difference by parts)
	
	For $x_0, x_1,y_0, y_1 \in \R$, 
	\begin{equation} \label{disc_diff_part}
		x_1y_1 - x_0 y_0 = x_1 \left(y_1 - y_0\right) + y_0 \left(x_1 - x_0\right).
	\end{equation}
	This can be generalised to hold over finite sums over $\R$. Let $x_i, y_i$ for $i \in \{1,...,N\}$, and let $\Delta_+(x_i) = x_{i+1} - x_i$. Then 
	
	\begin{equation} \label{discreteDiffPart}
		x_{N}y_{N} - x_{1}y_{1} = \sum_{i = 1}^{N-1} x_i \Delta_+(y_i) + \sum_{i = 1}^{N-1} y_{i+1}\Delta_+(x_i).
	\end{equation}
	The last formula can be seen from expanding the terms on the right, which yields a telescoping sum. 
\end{formula}

$\mathscr{P} = \{\{x_0, ..., x_N\} \subset \R$ for $N \in \mathscr{N}$ | $x_0 < x_1 < ... < x_N \}$ is the set of ordered partition of $\R$. For $p \in \mathscr{P}$, define
\begin{equation}
    p(f) = \sum_{i=0}^{\vert p\vert -1} |f(x_{i+1}) - f(x_i)|,
\end{equation}
where $\vert p \vert$ denotes the cardinality of $p$. 
\begin{equation}
    TV(f) := \sup_{p \in \mathscr{P}} p(f),
\end{equation}
\end{definition}
is the total variation of $f$. If supremum does not exist, we say that $TV(f) = \infty$. The space of functions with bounded total variation is denoted $BV(\R)$. This notion of BV can has several extensions to several dimensions. 

\begin{equation}
TV(f)     
\end{equation}

\section{Mollifier}
A smooth approximation to the identity, or mollifier, is a useful tool for creating smooth approximations. A mollifier can be defined as a function which satisfies

\begin{align}
\phi \ \in \mathscr{C}^{\infty}(\mathscr{R}^n) \\
\phi \geq 0 
\text{sup}(\phi) := 
\end{align}

forklar hvorfor antakelsene er rimelige. hva veivesenet gjør.

\section{Differentiation Identities and Calculus}
A useful result in the manipulation of differences is splitting of the sum. For any pair of indexed quantities, $x_0, x_1,y_0, y_1$, 
\begin{equation} \label{disc_diff_part}
    x_1y_1 - x_0 y_0 = x_1 \left(y_1 - y_0\right) + y_0 \left(x_1 - x_0\right).
\end{equation}
This can be generalised to hold over arbitrary sums. Let $x_i, y_i$ for $i \in \{1,...,N\}$, and let $\Delta_+(x_i) = x_{i+1} - x_i$. Then 

\begin{equation} \label{discreteDiffPart}
    x_{N}y_{N} - x_{1}y_{1} = \sum_{i = 1}^{N-1} x_i \Delta_+(y_i) + \sum_{i = 1}^{N-1} y_{i+1}\Delta_+(x_i).
\end{equation}
Expanding the terms on the right yields a telescoping sum. 

 CITE RUDIN.  The integral can be understood in Lebesgue sense. 

The Mean-Value formula is a fundamental result in Calculus, and goes as follows. 

\begin{}


Definition of a metric

inverse function theorem

the fundamental theorem of calculus

Proof: 

FUBINIS THEOREM 

In the following I will add a technical result that will prove useful in the discussion of the weak entropy solution

FAST L1 convergers implies convergence almost everywhere and almost uniformly convergernce. 

\begin{}

We can always pick a subsequence to obtain fast L1 convergence, and hence L1 convergence almost everywhere. 
Note that it does not hold that L1 converegnce implies almost everywhere convergence. But we can always find a subsequence which converges. 

List of assumptions: 
- k is C2, constant outside of of some ball of finite radius. The total variation of k and k' are bounded. k in L^\infty 
- V is bi-lipschitz, with the same things holding as in the article. 
- 


the mean value formula 

for example
Differentiation under the integral sign is 

The corresponding continuous variant is given a

% \begin{cases}
% \f^{(k)}(x)
% \end{cases}




$\mathscr{C}$
smooth with compact support,and therefore their convolution with any element function $f \in \$L_loc(\Omega)$A useful tool for function smoothing and approximation are the mollifier class of functions. The standard mollifier, 


Hysteresis?

\section{On the microscopic and macroscopic models}



The microscopic model is a collection of models that base around the fact that
\begin{align}
    \overset{z_{i-1/2}}{\cdot}_{i-1/2} = V\left(\frac{l}{z_{i+1/2} - z_{i-1/2}}\right)
\end{align}
was first introduced by pipes [1953], a modification of which will be considered in this assignment. Other variants can be introduced as well. The second order models 

Operate on different kinds of aggregation levels. 

Microscopic models 
From traffic and pedestrian follow-the-leader models with reaction time to first order convection-diffusion flow models:  - "Follow the leader models have many variants. One could consider accelration functions which". 

Macroscopid models differ from the microscopic model in that the dynamical quantities of interest are locally aggregated. [Martin Trieberg] Instead of focusing on eavh vehicle, one instead consideres traffic density $\rho(x,t)$, the flow density $Q(x,t)$, the mean speed $V(x,t)$. 

In physics, many quantities can be described aptly within the framework of conservation laws. A differential conservation law is an constitutive equation for a system which essentially states that for field quantities, such as density and density flux.  hyperoblic conservation law, or a Transport phenomena.  
\begin{equation} \label{conservation_law}
    \frac{\partial \rho }{\partial t } + \text{div}(f) = 0
\end{equation}
The equation states that in some fixed region of space, the net change in mass in any fixed region of space must cancel the net flow of mass across the boundary. 
The macroscopic traffic models are precisely those cast in the form of \eqref{conservation_law}. The model we will consider in are the celebrated LWR-model

begin{equation} 

\label{conservation_law}
    \frac{\partial \rho }{\partial t } + \text{\rho v\left(\rho\right)}(f) = 0
\end{equation}

mass across the region, modeled respectively by the first and second term. 


The variables are surpressed
Thus, macroscopic models are able to describe.  

My model: 
- One of the simplest approach is a speed model solely based on the spacing,firstly proposed by Pipes

Macroscopic models: 

The fundamental diagram is concave

Our case corresponds to congested traffic?

LWR refers to a whole
class of models.

Why macroscopic models are useful. 
    - Can describe collective phenomenon
    - only interested in macroscopic quantities 
    - Computation time of simulation is critical. 
More can be found in Martin Treiber • Arne Kesting, Traffic Flow dynamics. 
\section{Notation}

\begin{tabular}{r c p{10cm} }
\toprule
$\rightrightarrows$ & $\triangleq$ & everywhere uniform convergence $i$\\
$f_j$ & $\triangleq$ & $f(z_j)$ for any function that takes an indexed arg\\
$\text{T.V.}_x(\cdot)$ & $\triangleq$ & The total variation in the $x-$ direction.\\
$\C^{0,1}(\Omega)$ & $\triangleq$ & The space of Lipschitz continuous functions over $\Omega$\\ 
$\Delta_+(x_i)$ & $\triangleq$ & $x_{i+1} - x_{i}$ The forward difference operator\\
$D_+(x_i)$ & $\triangleq$ & $\frac{x_{i+1} - x_{i}}{l}$ A forward difference operator, that divide on the car length\\  
$D_+^z(x_i)$ & $\triangleq$ & $\frac{x_{i+1} - x_{i}}{z_{i+1/2}-z_{i-1/2}}$ A forward difference operator.\  
\multicolumn{3}{c}{}\\
\multicolumn{3}{c}{\underline{Decision Variables}}\\
\multicolumn{3}{c}{}\\
$y_f$ & $=$ & \(\begin{cases}
1,  & \text{if Supplier located at site $f$ is open} \\
0,  & \text{otherwise} \end{cases}\)\\
\bottomrule
\end{tabular}


Over the years, several thesis templates for \LaTeX{} have been developed by different groups at NTNU. Typically, there have been local templates for given study programmes, or different templates for the different study levels – bachelor, master, and \acrshort{phd}.\footnote{see, e.g., \url{https://github.com/COPCSE-NTNU/bachelor-thesis-NTNU} and \url{https://github.com/COPCSE-NTNU/master-theses-NTNU}}

Based on this experience, the \acrfull{CoPCSE}\footnote{\url{https://www.ntnu.no/wiki/display/copcse/Community+of+Practice+in+Computer+Science+Education+Home}} is hereby offering a template that should in principle be applicable for theses at all study levels. It is closely based on the standard \LaTeX{} \texttt{report} document class as well as previous thesis templates. Since the central regulations for thesis design have been relaxed – at least for some of the historical university colleges now part of NTNU – the template has been simlified and put closer to the default \LaTeX{} look and feel.

The purpose of the present document is threefold. It should serve (i) as a description of the document class, (ii) as an example of how to use it, and (iii) as a thesis template.

